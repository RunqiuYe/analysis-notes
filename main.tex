\documentclass[a4paper]{article}
\usepackage{newpxtext}
\usepackage{parskip}

\def\npart {}
\def\nterm {Spring}
\def\nyear {2025}
\def\nlecturer {Ian Tice}
\def\ncourse {Mathematical Studies of Analysis}

\makeatletter
\ifx \nauthor\undefined
  \def\nauthor{Runqiu Ye}
\else
\fi

\author{Notes taken by \nauthor\vspace{5pt}\\ 
Carnegie Mellon University}
\date{\nterm\ \nyear}

\usepackage{alltt}
\usepackage{amsfonts}
\usepackage{amsmath}
\usepackage{amssymb}
\usepackage{amsthm}
\usepackage{booktabs}
\usepackage{caption}
\usepackage{enumitem}
\usepackage{fancyhdr}
\usepackage{graphicx}
\usepackage{mathdots}
\usepackage{mathtools}
\usepackage{microtype}
\usepackage{multirow}
\usepackage{pdflscape}
\usepackage{pgfplots}
\usepackage{siunitx}
\usepackage{slashed}
\usepackage{tabularx}
\usepackage{tikz}
\usepackage{tkz-euclide}
\usepackage[normalem]{ulem}
\usepackage[all]{xy}
\usepackage{imakeidx}
\usepackage[includehead,includefoot, heightrounded,  left=1.0in, top=1.6cm, bottom=2.4cm, right=1.0in]{geometry}
\usepackage{mathrsfs}

\makeindex[intoc, title=Index]
\indexsetup{othercode={\lhead{\emph{Index}}}}

\ifx \nextra \undefined
  \usepackage[pdftex,
    hidelinks,
    pdfauthor={Dexter Chua},
    pdfsubject={\ncourse},
    pdftitle={\ncourse},
  pdfkeywords={Cambridge Mathematics Maths Math \nterm\ \nyear\ \ncourse}]{hyperref}
  \title{\ncourse}
\else
  \usepackage[pdftex,
    hidelinks,
    pdfauthor={Dexter Chua},
    pdfsubject={Cambridge Maths Notes: \ncourse\ (\nextra)},
    pdftitle={\ncourse\ (\nextra)},
  pdfkeywords={Cambridge Mathematics Maths Math \nterm\ \nyear\ \ncourse\ \nextra}]{hyperref}

  \title{\ncourse \\ {\Large \nextra}}
  \renewcommand\printindex{}
\fi

\pgfplotsset{compat=1.12}

\pagestyle{fancyplain}
\ifx \ncoursehead \undefined
\def\ncoursehead{\ncourse}
\fi

\lhead{\emph{\nouppercase{\leftmark}}}
\ifx \nextra \undefined
  \rhead{
    \ifnum\thepage=1
    \else
      \ncoursehead
    \fi}
\else
  \rhead{
    \ifnum\thepage=1
    \else
      \ncoursehead \ (\nextra)
    \fi}
\fi
\usetikzlibrary{arrows.meta}
\usetikzlibrary{decorations.markings}
\usetikzlibrary{decorations.pathmorphing}
\usetikzlibrary{positioning}
\usetikzlibrary{fadings}
\usetikzlibrary{intersections}
\usetikzlibrary{cd}

\newcommand*{\Cdot}{{\raisebox{-0.25ex}{\scalebox{1.5}{$\cdot$}}}}
\newcommand {\pd}[2][ ]{
  \ifx #1 { }
    \frac{\partial}{\partial #2}
  \else
    \frac{\partial^{#1}}{\partial #2^{#1}}
  \fi
}
\ifx \nhtml \undefined
\else
  \renewcommand\printindex{}
  \DisableLigatures[f]{family = *}
  \let\Contentsline\contentsline
  \renewcommand\contentsline[3]{\Contentsline{#1}{#2}{}}
  \renewcommand{\@dotsep}{10000}
  \newlength\currentparindent
  \setlength\currentparindent\parindent

  \newcommand\@minipagerestore{\setlength{\parindent}{\currentparindent}}
  \usepackage[active,tightpage,pdftex]{preview}
  \renewcommand{\PreviewBorder}{0.1cm}

  \newenvironment{stretchpage}%
  {\begin{preview}\begin{minipage}{\hsize}}%
    {\end{minipage}\end{preview}}
  \AtBeginDocument{\begin{stretchpage}}
  \AtEndDocument{\end{stretchpage}}

  \newcommand{\@@newpage}{\end{stretchpage}\begin{stretchpage}}

  \let\@real@section\section
  \renewcommand{\section}{\@@newpage\@real@section}
  \let\@real@subsection\subsection
  \renewcommand{\subsection}{\@ifstar{\@real@subsection*}{\@@newpage\@real@subsection}}
\fi
\ifx \ntrim \undefined
\else
  \usepackage{geometry}
  \geometry{
    papersize={379pt, 699pt},
    textwidth=345pt,
    textheight=596pt,
    left=17pt,
    top=54pt,
    right=17pt
  }
\fi

\ifx \nisofficial \undefined
\let\@real@maketitle\maketitle
\renewcommand{\maketitle}{\@real@maketitle}
\else
\fi

% Theorems
\theoremstyle{definition}
\newtheorem*{aim}{Aim}
\newtheorem*{axiom}{Axiom}
\newtheorem*{claim}{Claim}
\newtheorem*{cor}{Corollary}
\newtheorem*{conjecture}{Conjecture}
\newtheorem*{defi}{Definition}
\newtheorem*{eg}{Example}
\newtheorem*{ex}{Exercise}
\newtheorem*{fact}{Fact}
\newtheorem*{law}{Law}
\newtheorem*{lemma}{Lemma}
\newtheorem*{notation}{Notation}
\newtheorem*{prop}{Proposition}
\newtheorem*{question}{Question}
\newtheorem*{rrule}{Rule}
\newtheorem*{thm}{Theorem}
\newtheorem*{assumption}{Assumption}

\newtheorem*{remark}{Remark}
\newtheorem*{warning}{Warning}
\newtheorem*{exercise}{Exercise}

\newtheorem{nthm}{Theorem}[section]
\newtheorem{nlemma}[nthm]{Lemma}
\newtheorem{nprop}[nthm]{Proposition}
\newtheorem{ncor}[nthm]{Corollary}


\renewcommand{\labelitemi}{--}
\renewcommand{\labelitemii}{$\circ$}
% \renewcommand{\labelenumi}{(\roman{*})}

\let\stdsection\section
\renewcommand\section{\newpage\stdsection}

\newcommand\qedsym{\hfill\ensuremath{\square}}
% Strike through
\def\st{\bgroup \ULdepth=-.55ex \ULset}



%%%%%%%%%%%%%%%%%%%%%%%%%
%%%%% Maths Symbols %%%%%
%%%%%%%%%%%%%%%%%%%%%%%%%
\newcommand{\cupinfn}{\bigcup_{n=0}^\infty}
\newcommand{\capinfn}{\bigcap_{n=0}^\infty}
\newcommand{\suminfn}{\sum_{n=0}^\infty}
\newcommand{\seqinfn}[1]{\left\{ #1 \right\}_{n=0}^\infty}
\newcommand{\cupinfk}{\bigcup_{k=0}^\infty}
\newcommand{\capinfk}{\bigcap_{k=0}^\infty}
\newcommand{\suminfk}{\sum_{k=0}^\infty}
\newcommand{\seqinfk}[1]{\left\{ #1 \right\}_{k=0}^\infty}
\newcommand{\cupinfm}{\bigcup_{m=0}^\infty}
\newcommand{\capinfm}{\bigcap_{m=0}^\infty}
\newcommand{\suminfm}{\sum_{m=0}^\infty}
\newcommand{\seqinfm}[1]{\left\{ #1 \right\}_{m=0}^\infty}

\newcommand{\cupn}{\bigcup_{n=0}}
\newcommand{\capn}{\bigcap_{n=0}}
\newcommand{\sumn}{\sum_{n=0}}
\newcommand{\seqn}[1]{\left\{ #1 \right\}_{n=0}}
\newcommand{\cupk}{\bigcup_{k=0}}
\newcommand{\capk}{\bigcap_{k=0}}
\newcommand{\sumk}{\sum_{k=0}}
\newcommand{\seqk}[1]{\left\{ #1 \right\}_{k=0}}
\newcommand{\cupm}{\bigcup_{m=0}}
\newcommand{\capm}{\bigcap_{m=0}}
\newcommand{\summ}{\sum_{m=0}}
\newcommand{\seqm}[1]{\left\{ #1 \right\}_{m=0}}

% Analysis
\newcommand{\om}{\mu^*}
\newcommand{\fs}{F_\sigma}
\newcommand{\gd}{G_\delta}
\newcommand{\cale}{\mathcal{E}}
\newcommand{\calf}{\mathcal{F}}
\newcommand{\mf}{\mathfrak{M}}
\newcommand{\nf}{\mathfrak{N}}
\newcommand{\bfk}{\mathfrak{B}}
\def\L{\mathcal{L}}
\def\H{\mathcal{H}}
\renewcommand{\P}{\mathcal{P}}
\DeclareMathOperator{\Rec}{Rec}
\DeclareMathOperator{\Cube}{Cube}
\DeclareMathOperator{\RCube}{RCube}
\newcommand{\RCuber}{\RCube_r}
\DeclareMathOperator{\Reco}{Rec^{\circ}}
\DeclareMathOperator{\Cubeo}{Cube^{\circ}}
\DeclareMathOperator{\RCubeo}{RCube^{\circ}}
\newcommand{\RCubeor}{\RCube^{\circ}_r}

% Matrix groups
\newcommand{\GL}{\mathrm{GL}}
\newcommand{\Or}{\mathrm{O}}
\newcommand{\PGL}{\mathrm{PGL}}
\newcommand{\PSL}{\mathrm{PSL}}
\newcommand{\PSO}{\mathrm{PSO}}
\newcommand{\PSU}{\mathrm{PSU}}
\newcommand{\SL}{\mathrm{SL}}
\newcommand{\SO}{\mathrm{SO}}
\newcommand{\Spin}{\mathrm{Spin}}
\newcommand{\Sp}{\mathrm{Sp}}
\newcommand{\SU}{\mathrm{SU}}
\newcommand{\U}{\mathrm{U}}
\newcommand{\Mat}{\mathrm{Mat}}

% Matrix algebras
\newcommand{\gl}{\mathfrak{gl}}
\newcommand{\ort}{\mathfrak{o}}
\newcommand{\so}{\mathfrak{so}}
\newcommand{\su}{\mathfrak{su}}
\newcommand{\uu}{\mathfrak{u}}
\renewcommand{\sl}{\mathfrak{sl}}

% Special sets
\newcommand{\C}{\mathbb{C}}
\newcommand{\CP}{\mathbb{CP}}
\newcommand{\GG}{\mathbb{G}}
\newcommand{\N}{\mathbb{N}}
\newcommand{\Q}{\mathbb{Q}}
\newcommand{\R}{\mathbb{R}}
\newcommand{\RP}{\mathbb{RP}}
\newcommand{\T}{\mathbb{T}}
\newcommand{\Z}{\mathbb{Z}}
% \renewcommand{\H}{\mathbb{H}}

% Brackets
\renewcommand{\bar}[1]{\overline{#1}}
\renewcommand{\tilde}[1]{\widetilde{#1}}
\renewcommand{\hat}[1]{\widehat{#1}}
\newcommand{\floor}[1]{\left\lfloor #1 \right\rfloor}
\newcommand{\ceil}[1]{\left\lceil #1 \right\rceil}
\newcommand{\abs}[1]{\left\lvert #1\right\rvert}
\newcommand{\bket}[1]{\left\lvert #1\right\rangle}
\newcommand{\brak}[1]{\left\langle #1 \right\rvert}
\newcommand{\braket}[2]{\left\langle #1, #2 \right\rangle}
\newcommand{\bra}{\langle}
\newcommand{\ket}{\rangle}
\newcommand{\norm}[1]{\left\lVert #1\right\rVert}
\newcommand{\normalorder}[1]{\mathop{:}\nolimits\!#1\!\mathop{:}\nolimits}
\newcommand{\tv}[1]{|#1|}
\renewcommand{\vec}[1]{\boldsymbol{\mathbf{#1}}}
\DeclareMathOperator{\curl}{curl}
\DeclareMathOperator{\diverge}{div}
\DeclareMathOperator{\dist}{dist}

% not-math
\newcommand{\bolds}[1]{{\bfseries #1}}
\newcommand{\cat}[1]{\mathsf{#1}}
\newcommand{\ph}{\,\cdot\,}
\newcommand{\term}[1]{\emph{#1}\index{#1}}
\newcommand{\phantomeq}{\hphantom{{}={}}}
% Probability
\DeclareMathOperator{\Bernoulli}{Bernoulli}
\DeclareMathOperator{\betaD}{beta}
\DeclareMathOperator{\bias}{bias}
\DeclareMathOperator{\binomial}{binomial}
\DeclareMathOperator{\corr}{corr}
\DeclareMathOperator{\cov}{cov}
\DeclareMathOperator{\gammaD}{gamma}
\DeclareMathOperator{\mse}{mse}
\DeclareMathOperator{\multinomial}{multinomial}
\DeclareMathOperator{\Poisson}{Poisson}
\DeclareMathOperator{\var}{var}
\newcommand{\E}{\mathbb{E}}
\newcommand{\Prob}{\mathbb{P}}

% Algebra
\DeclareMathOperator{\adj}{adj}
\DeclareMathOperator{\Ann}{Ann}
\DeclareMathOperator{\Aut}{Aut}
\DeclareMathOperator{\Char}{char}
\DeclareMathOperator{\disc}{disc}
\DeclareMathOperator{\dom}{dom}
\DeclareMathOperator{\fix}{fix}
\DeclareMathOperator{\Hom}{Hom}
\DeclareMathOperator{\id}{id}
\DeclareMathOperator{\image}{image}
\DeclareMathOperator{\im}{im}
\DeclareMathOperator{\tr}{tr}
\DeclareMathOperator{\Tr}{Tr}
\newcommand{\Bilin}{\mathrm{Bilin}}
\newcommand{\Frob}{\mathrm{Frob}}

% Others
\newcommand\ad{\mathrm{ad}}
\newcommand\Art{\mathrm{Art}}
\newcommand{\B}{\mathcal{B}}
\newcommand{\cU}{\mathcal{U}}
\newcommand{\Der}{\mathrm{Der}}
\newcommand{\D}{\mathrm{D}}
\newcommand{\dR}{\mathrm{dR}}
\newcommand{\exterior}{\mathchoice{{\textstyle\bigwedge}}{{\bigwedge}}{{\textstyle\wedge}}{{\scriptstyle\wedge}}}
\newcommand{\F}{\mathbb{F}}
\newcommand{\G}{\mathcal{G}}
\newcommand{\Gr}{\mathrm{Gr}}
\newcommand{\haut}{\mathrm{ht}}
\newcommand{\Hol}{\mathrm{Hol}}
\newcommand{\hol}{\mathfrak{hol}}
\newcommand{\Id}{\mathrm{Id}}
\newcommand{\lie}[1]{\mathfrak{#1}}
\newcommand{\op}{\mathrm{op}}
\newcommand{\Oc}{\mathcal{O}}
\newcommand{\pr}{\mathrm{pr}}
\newcommand{\Ps}{\mathcal{P}}
\newcommand{\pt}{\mathrm{pt}}
\newcommand{\qeq}{\mathrel{``{=}"}}
\newcommand{\Rs}{\mathcal{R}}
\newcommand{\Vect}{\mathrm{Vect}}
\newcommand{\wsto}{\stackrel{\mathrm{w}^*}{\to}}
\newcommand{\wt}{\mathrm{wt}}
\newcommand{\wto}{\stackrel{\mathrm{w}}{\to}}
\renewcommand{\d}{\mathrm{d}}
\renewcommand{\Prob}{\mathbb{P}}
%\renewcommand{\F}{\mathcal{F}}
\def\st{\;\vert\;}
\newcommand{\nab}{\nabla}
\renewcommand{\varnothing}{\emptyset}
\newcommand{\nsubset}{\not\subset}
\renewcommand{\epsilon}{\varepsilon}
\newcommand{\ep}{\varepsilon}
\renewcommand{\phi}{\varphi}


\let\Im\relax
\let\Re\relax

\DeclareMathOperator{\RHS}{RHS}
\DeclareMathOperator{\LHS}{LHS}
\DeclareMathOperator{\step}{Step}
\DeclareMathOperator{\reg}{Reg}
\DeclareMathOperator{\ran}{range}
\DeclareMathOperator{\area}{area}
\DeclareMathOperator{\card}{card}
\DeclareMathOperator{\ccl}{ccl}
\DeclareMathOperator{\ch}{ch}
\DeclareMathOperator{\cl}{cl}
\DeclareMathOperator{\cls}{\overline{\mathrm{span}}}
\DeclareMathOperator{\coker}{coker}
\DeclareMathOperator{\conv}{conv}
\DeclareMathOperator{\cosec}{cosec}
\DeclareMathOperator{\cosech}{cosech}
\DeclareMathOperator{\covol}{covol}
\DeclareMathOperator{\diag}{diag}
\DeclareMathOperator{\diam}{diam}
\DeclareMathOperator{\Diff}{Diff}
\DeclareMathOperator{\End}{End}
\DeclareMathOperator{\energy}{energy}
\DeclareMathOperator{\erfc}{erfc}
\DeclareMathOperator{\erf}{erf}
\DeclareMathOperator*{\esssup}{ess\,sup}
\DeclareMathOperator{\ev}{ev}
\DeclareMathOperator{\Ext}{Ext}
\DeclareMathOperator{\fst}{fst}
\DeclareMathOperator{\Fit}{Fit}
\DeclareMathOperator{\Frac}{Frac}
\DeclareMathOperator{\Gal}{Gal}
\DeclareMathOperator{\gr}{gr}
\DeclareMathOperator{\hcf}{hcf}
\DeclareMathOperator{\Im}{Im}
\DeclareMathOperator{\Ind}{Ind}
\DeclareMathOperator{\Int}{Int}
\DeclareMathOperator{\Isom}{Isom}
\DeclareMathOperator{\lcm}{lcm}
\DeclareMathOperator{\length}{length}
\DeclareMathOperator{\Lie}{Lie}
\DeclareMathOperator{\like}{like}
\DeclareMathOperator{\Lk}{Lk}
\DeclareMathOperator{\Maps}{Maps}
\DeclareMathOperator{\orb}{orb}
\DeclareMathOperator{\ord}{ord}
\DeclareMathOperator{\otp}{otp}
\DeclareMathOperator{\poly}{poly}
\DeclareMathOperator{\rank}{rank}
\DeclareMathOperator{\rel}{rel}
\DeclareMathOperator{\Rad}{Rad}
\DeclareMathOperator{\Re}{Re}
\DeclareMathOperator*{\res}{res}
\DeclareMathOperator{\Res}{Res}
\DeclareMathOperator{\Ric}{Ric}
\DeclareMathOperator{\rk}{rk}
\DeclareMathOperator{\Rees}{Rees}
\DeclareMathOperator{\Root}{Root}
\DeclareMathOperator{\sech}{sech}
\DeclareMathOperator{\sgn}{sgn}
\DeclareMathOperator{\snd}{snd}
\DeclareMathOperator{\Spec}{Spec}
\DeclareMathOperator{\spn}{span}
\DeclareMathOperator{\stab}{stab}
\DeclareMathOperator{\St}{St}
\DeclareMathOperator{\supp}{supp}
\DeclareMathOperator{\Syl}{Syl}
\DeclareMathOperator{\Sym}{Sym}
\DeclareMathOperator{\vol}{vol}

\pgfarrowsdeclarecombine{twolatex'}{twolatex'}{latex'}{latex'}{latex'}{latex'}
\tikzset{->/.style = {decoration={markings,
                                  mark=at position 1 with {\arrow[scale=2]{latex'}}},
                      postaction={decorate}}}
\tikzset{<-/.style = {decoration={markings,
                                  mark=at position 0 with {\arrowreversed[scale=2]{latex'}}},
                      postaction={decorate}}}
\tikzset{<->/.style = {decoration={markings,
                                   mark=at position 0 with {\arrowreversed[scale=2]{latex'}},
                                   mark=at position 1 with {\arrow[scale=2]{latex'}}},
                       postaction={decorate}}}
\tikzset{->-/.style = {decoration={markings,
                                   mark=at position #1 with {\arrow[scale=2]{latex'}}},
                       postaction={decorate}}}
\tikzset{-<-/.style = {decoration={markings,
                                   mark=at position #1 with {\arrowreversed[scale=2]{latex'}}},
                       postaction={decorate}}}
\tikzset{->>/.style = {decoration={markings,
                                  mark=at position 1 with {\arrow[scale=2]{latex'}}},
                      postaction={decorate}}}
\tikzset{<<-/.style = {decoration={markings,
                                  mark=at position 0 with {\arrowreversed[scale=2]{twolatex'}}},
                      postaction={decorate}}}
\tikzset{<<->>/.style = {decoration={markings,
                                   mark=at position 0 with {\arrowreversed[scale=2]{twolatex'}},
                                   mark=at position 1 with {\arrow[scale=2]{twolatex'}}},
                       postaction={decorate}}}
\tikzset{->>-/.style = {decoration={markings,
                                   mark=at position #1 with {\arrow[scale=2]{twolatex'}}},
                       postaction={decorate}}}
\tikzset{-<<-/.style = {decoration={markings,
                                   mark=at position #1 with {\arrowreversed[scale=2]{twolatex'}}},
                       postaction={decorate}}}

\tikzset{circ/.style = {fill, circle, inner sep = 0, minimum size = 3}}
\tikzset{scirc/.style = {fill, circle, inner sep = 0, minimum size = 1.5}}
\tikzset{mstate/.style={circle, draw, blue, text=black, minimum width=0.7cm}}

\tikzset{eqpic/.style={baseline={([yshift=-.5ex]current bounding box.center)}}}
\tikzset{commutative diagrams/.cd,cdmap/.style={/tikz/column 1/.append style={anchor=base east},/tikz/column 2/.append style={anchor=base west},row sep=tiny}}

\definecolor{mblue}{rgb}{0.2, 0.3, 0.8}
\definecolor{morange}{rgb}{1, 0.5, 0}
\definecolor{mgreen}{rgb}{0.1, 0.4, 0.2}
\definecolor{mred}{rgb}{0.5, 0, 0}

\def\drawcirculararc(#1,#2)(#3,#4)(#5,#6){%
    \pgfmathsetmacro\cA{(#1*#1+#2*#2-#3*#3-#4*#4)/2}%
    \pgfmathsetmacro\cB{(#1*#1+#2*#2-#5*#5-#6*#6)/2}%
    \pgfmathsetmacro\cy{(\cB*(#1-#3)-\cA*(#1-#5))/%
                        ((#2-#6)*(#1-#3)-(#2-#4)*(#1-#5))}%
    \pgfmathsetmacro\cx{(\cA-\cy*(#2-#4))/(#1-#3)}%
    \pgfmathsetmacro\cr{sqrt((#1-\cx)*(#1-\cx)+(#2-\cy)*(#2-\cy))}%
    \pgfmathsetmacro\cA{atan2(#2-\cy,#1-\cx)}%
    \pgfmathsetmacro\cB{atan2(#6-\cy,#5-\cx)}%
    \pgfmathparse{\cB<\cA}%
    \ifnum\pgfmathresult=1
        \pgfmathsetmacro\cB{\cB+360}%
    \fi
    \draw (#1,#2) arc (\cA:\cB:\cr);%
}
\newcommand\getCoord[3]{\newdimen{#1}\newdimen{#2}\pgfextractx{#1}{\pgfpointanchor{#3}{center}}\pgfextracty{#2}{\pgfpointanchor{#3}{center}}}

\newcommand\qedshift{\vspace{-17pt}}
\newcommand\fakeqed{\pushQED{\qed}\qedhere}

\def\Xint#1{\mathchoice
   {\XXint\displaystyle\textstyle{#1}}%
   {\XXint\textstyle\scriptstyle{#1}}%
   {\XXint\scriptstyle\scriptscriptstyle{#1}}%
   {\XXint\scriptscriptstyle\scriptscriptstyle{#1}}%
   \!\int}
\def\XXint#1#2#3{{\setbox0=\hbox{$#1{#2#3}{\int}$}
     \vcenter{\hbox{$#2#3$}}\kern-.5\wd0}}
\def\ddashint{\Xint=}
\def\dashint{\Xint-}

\newcommand\separator{{\centering\rule{2cm}{0.2pt}\vspace{2pt}\par}}

\newenvironment{own}{\color{gray!70!black}}{}

\newcommand\makecenter[1]{\raisebox{-0.5\height}{#1}}

\mathchardef\mdash="2D

\newenvironment{significant}{\begin{center}\begin{minipage}{0.9\textwidth}\centering\em}{\end{minipage}\end{center}}
\DeclareRobustCommand{\rvdots}{%
  \vbox{
    \baselineskip4\p@\lineskiplimit\z@
    \kern-\p@
    \hbox{.}\hbox{.}\hbox{.}
  }}
\DeclareRobustCommand\tph[3]{{\texorpdfstring{#1}{#2}}}
\makeatother


\begin{document}
\maketitle

\tableofcontents

\section{Advanced topics in metric space theory}

\subsection{Baire category}

\begin{defi}
Let $X$ be a metric space.  
\begin{enumerate}
 \item We say that $E \subset X$ is nowhere dense if 
 $(\bar{E})^\circ = \varnothing$.
 \item We say that $E \subset X$ is meager in $X$ if 
\begin{equation*}
 E = \bigcup_{\alpha \in A} E_\alpha,
\end{equation*}
where $A$ is a countable set and $E_\alpha \subset X$ 
is nowhere dense for every $\alpha \in A$.
\end{enumerate}
\end{defi}

\begin{thm}
Prove that the following are equivalent for  
$E \subset X$:
\begin{enumerate}
 \item $E$ is nowhere dense
 \item $\bar{E}$ is nowhere dense
 \item $(\bar{E})^c$ is open and dense in $X$.
\end{enumerate}
\end{thm}

\begin{proof}
  (a) $\implies$ (b). Suppose $E$ is nowhere dense, then 
  $(\bar{E})^\circ = \emptyset$. Note that the closure 
  of $\bar{E}$ is just $\bar{E}$ itself. It follows that 
  $\bar{E}$ is also nowhere dense. 
  
  (b) $\implies$ (c). Suppose $\bar{E}$ is nowhere dense.
  Note that $\bar{E}$ is closed, so $(\bar{E})^c$ is open.
  Let $x \in X$ be arbitrary. Since $\bar{E}$ is nowhere dense,
  $x \notin (\bar{E})^\circ$. This implies that for arbitrary 
  $\epsilon > 0$, we have $B(x, \epsilon) \nsubset \bar{E}$.
  This is equivalent to $B(x, \epsilon) \cap (\bar{E})^c \neq 
  \emptyset$. Hence, $(\bar{E})^c$ is dense in $X$. 

  (c) $\implies$ (a). Suppose $(\bar{E})^c$ is dense in $X$. 
  Let $x \in X$ and $\epsilon > 0$ be arbitrary. It follows 
  that $B(x, \epsilon) \cap (\bar{E})^c \neq \emptyset$. 
  This is equivalent to $B(x, \epsilon) \nsubset \bar{E}$. 
  Therefore, $(\bar{E})^\circ = \emptyset$ and $E$ is nowhere 
  dense.
\end{proof}

\begin{thm}[Baire category thorem]
Let $X$ be a complete metric space.  Suppose that for each 
$n \in \N$, $U_n \subset X$ is open and dense in $X$. 
Prove that $\bigcap_{n=0}^\infty U_n$ is dense in $X$.  
Hint: use the shrinking closed set property.
\end{thm}

\begin{proof}
  Consider any $x \in X$ and arbitrary $\epsilon > 0$, it
  suffices to show that $U_n \cap B(x, \epsilon)
  \neq \emptyset$ for each $n \in \N$. 
  Now inductively choosing a sequence 
  $x_i \in X$ and $\epsilon_i > 0$ such that 
  for each $i \in \N$, $B[x_i, \epsilon_i] \subset U_i$,
  $B[x_{i+1}, \epsilon_i] \subset B[x_i, \epsilon_i]
  \subset B(x, \epsilon)$, and $\epsilon_i < 2^{-i} \epsilon$.
  
  Since $U_0$ is dense in $X$,
  $B(x, \epsilon) \cap U_0 \neq \emptyset$.
  Note that both $U_0$ and $B(x, \epsilon)$ are open, so
  we can choose $x_0 \in B(x, \epsilon) \cap U_0$ 
  and $\epsilon_0 > 0$ so small 
  that $B[x_0, \epsilon_0] \subset B(x, \epsilon) \cap U_0$
  and $\epsilon_0 < \epsilon$. 
  Now suppose 
  for $0 \leq i \leq n$, we have chosen $x_i \in X$ 
  and $\epsilon_i > 0$ such that 
  $B[x_i, \epsilon_i] \subset U_i$ 
  and $\epsilon_i < 2^{-i} \epsilon$ 
  for all $0 \leq i \leq n$,
  $B[x_{i+1}, \epsilon_{i+1}] \subset 
  B[x_i, \epsilon_i]$ for all $0 \leq i < n$.
  Since $U_{n+1}$ is dense in $X$, $B(x_n, \epsilon_n)
  \cap U_{n+1} \neq \emptyset$. Note also both $U_{n+1}$ 
  and $B(x_n, \epsilon_n)$ are open.
  Therefore, choose 
  $x_{n+1} \in B(x_n, \epsilon_n) \cap U_{n+1}$ and 
  $\epsilon_{n+1} > 0$ so small that 
  $B[x_{n+1}, \epsilon_{n+1}] \subset B(x_n, \epsilon_n) 
  \cap U_{n+1}$ and $\epsilon_{n+1} < \frac{\epsilon_n}{2}$. 
  It follows that 
  $B[x_{n+1}, \epsilon_{n+1}] \subset U_{n+1}$ and 
  $B[x_{n+1}, \epsilon_{n+1}] \subset B[x_n, \epsilon_n]
  \subset B(x, \epsilon)$. Also, 
  $\epsilon < \frac{\epsilon_n}{2} < 2^{-n-1} \epsilon$.
  Now we have successfully constructing the desired sequence.
  
  Since $X$ is complete, $\bigcap_{n=0}^\infty B[x_n, \epsilon_n]
  = \left\{ z \right\}$ for some $z \in X$. Note that 
  for each $n$, we have $z \in B[x_n, \epsilon_n] \subset U_n$. 
  Also,
  $z \in B[x_n, \epsilon_n] \subset B(x, \epsilon)$. 
  Therefore, $z \in U_n \cap B(x, \epsilon)$ for each $n \in \N$
  and $\bigcap_{n=0}^\infty U_n$ is dense in $X$.
\end{proof}

\begin{remark}
    An equivalent statement of the theorem is the following: \\
    Let $X$ be a complete metric space and $\{C_n\}$ a countable 
    collection of closed subsets of $X$ such that $X = 
    \bigcup_{n \in \N} C_n $. Then at least one of the $C_n$ 
    contains an open ball.
\end{remark}

\subsection{Open Mapping Theorem}

\subsubsection*{Linear surjections}

\begin{thm}[Open mapping theorem]
Let $X,Y$ be Banach spaces over a common field and assume that 
$T \in \L(X;Y)$.  Prove that the following are equivalent.
\begin{enumerate}
 \item $T$ is surjective.
 
 \item There exists $\delta >0$ such that $B_Y(0,\delta) \subset 
 \overline{T(B_X(0,1))}$.
 
 \item For every $\epsilon >0$ there exists $\delta >0$ 
 such that $B_Y(0,\delta) \subset T(B_X(0,\ep))$.
 
 \item $T$ is an open map: if $U\subset X$ is open, then 
 $T(U) \subset Y$ is open.

  \item There exists $C \ge 0$ such that for each $y \in Y$ 
  there exists $x \in X$ such that $Tx=y$ and 
\begin{equation*}
 \norm{x}_X \le C \norm{y}_Y.
\end{equation*}
\end{enumerate}
HINT: Prove that  $(1) \implies (2) \implies (3) 
\implies (4) \implies (5)  \implies (1)$, keeping 
in mind the following suggestions.  
\begin{enumerate}
 \item For (1) $\implies$ (2): Study the sets $C_n = 
 \overline{T(B_X(0,n))} \subset Y$ for $n \ge 1$.
 \item For (2) $\implies$ (3):  Prove that 
 $\overline{T(B_X(0,1)  )} \subset T(B_X(0,3))$ 
 by considering $y \in\overline{T(B_X(0,1)  )}$ and  
 inductively constructing $\{x_j\}_{j=0}^\infty \subset X$
such that $\norm{x_j}_X < 2^{-j}$ and 
$y - \sum_{j=0}^m T x_j \in B_Y(0,2^{-m-1} R)$  
for all $m \in \N$.
\end{enumerate}
\end{thm}

\begin{proof}
  (1) $\implies$ (2). Following the hint, for $n \geq 1$ let 
  $C_n = \bar{T(B_X(0, n))}$. Then each of the $C_n$ are closed.
  Since $T$ is surjective, $Y = \bigcup_{n=1}^\infty C_n$. 
  Suppose for contradiction that each $C_n$ are nowhere dense.
  It then follows that $C_n^c$ are dense in $Y$. By Baire Category
  Theorem, $\bigcap_{n=1}^\infty C_n^c$ is dense in $Y$. However,
  $\bigcap_{n=1}^\infty C_n^c = \left( \bigcup_{n=1}^\infty C_n \right)^c 
  = \emptyset$, a contradiction. Therefore, at least one $C_n$
  is not nowhere dense. That is, there exists some $n \geq 1$, 
  $\bar{T(B_X(0, n))}$ contains an open ball. However, 
  this is the same set as $n \bar{T(B_X(0, 1))}$. Therefore, 
  $\bar{T(B_X(0, 1))}$ contains an open ball $B_Y(y_0, 4 r)$
  for some $y_0 \in Y$ and $r > 0$.

  Let $y_1 = T x_1$ for some $x_1 \in B_Y(0,1)$ such that 
  $\norm{y_0 - y_1} < 2 r$. It follows that 
  $B_Y(y_1, 2 r) \subset B_Y(y_0, 4 r) \subset \bar{T(B_X(0,1))}$. 
  For any $y \in Y$ such that 
  $\norm{y} < r$, we have 
  \[
  y = -\frac{1}{2} y_1 + \frac{1}{2} (2 y + y_1) = 
  - T \left( \frac{x_1}{2} \right) + \frac{1}{2} (2 y + y_1).
  \]
  However, notice that 
  \[
  \frac{1}{2} (2 y + y_1) \subset \frac{1}{2} B_Y(y_1, 2r) 
  \subset \frac{1}{2} \bar{T(B_X(0, 1))} 
  = \bar{T(B_X(0, \tfrac{1}{2}))} .
  \]
  It follows that 
  \[
  y = - T \left( \frac{x_1}{2} \right) + \frac{1}{2} (2 y + y_1)
  \in - T \left( \frac{x_1}{2} \right) + \bar{T(B_X(0, \tfrac{1}{2}))}.
  \]
  Note that $- T(\frac{x_1}{2}) \in T(B_X(0, \frac{1}{2}))$. Therefore, 
  $y \in \bar{T(B_X(0, 1))}$. Since $y$ is arbitrary with 
  $\norm{y} < r$, we have $B_Y(0, r) \subset \bar{T(B_X(0,1))}$.

  (2) $\implies$ (3). Following the hint, we first show 
  $\bar{T(B_X(0, 1))} \subset T(B_X(0, 3))$.
  By assumption, we have 
  $B_Y(0, R) \subset \bar{T(B_X(0, 1))}$ for some $R > 0$. 
  It follows from homogeneity that for each $m \in \N$, we have 
  \[
    2^{-m} B_Y(0, R) = B_Y(0, 2^{-m} R) \subset 
    2^{-m} \bar{T(B_X(0, 1))} = \bar{T(B_X(0, 2^{-m}))}.
  \]
  Let 
  $y \in \bar{T(B_X(0, 1))}$ and pick $x_0 \in X$ with 
  $\norm{x} < 1$ such that $\norm{y - Tx} < 2^{-1} R$. Now 
  suppose we have chosen $x_j$ for $0 \leq j \leq m$ such that 
  $\norm{x_j} < 2^{-j}$ and $y - \sum_{j=0}^m T x_j \in B_Y(0, 
  2^{-m - 1} R) $ for all $m \in \N$. By the inclusion above, 
  we can pick $x_{m+1} \in X$ with $\norm{x_{m+1}} 
  < 2^{-m-1}$ such that
  \[
  \norm{y - \sum_{j=0}^m T x_j - T x_{m+1}} 
  = \norm{y - \sum_{j=0}^{m+1} T x_j} < 2^{-m-2} R.
  \]
  Therefore, $y - \sum_{j=0}^{m+1} T x_j \in B_Y(0, 2^{-m-2}) R$.
  This completes the inductive construction, and we have
  found a sequence $\left\{ x_j \right\}$ such that 
  $\norm{x_j} < 2^{-j}$ and 
  $y - \sum_{j=0}^m T x_j \in B_Y (0, 2^{-m-1} R)$
  for each $m \in \N$. Note that 
  \[
  \sum_{j=0}^\infty \norm{x_j} \leq \sum_{j=0}^{\infty} 2^{-j} 
  = 2,
  \]
  so $\sum_{j=0}^\infty x_j$ converges absolutely. Since $X$ 
  is Banach, $\sum_{j=0}^\infty x_j$ converges
  to some $x \in X$ with $\norm{x} \leq 2$. Also, since 
  $y - \sum_{j=0}^m T x_j \in B_Y(0, 2^{-m-1}R)$, taking the
  limit where $m$ approaches infinity we obtain 
  \[
  y = \sum_{j=0}^\infty T x_j = T \left( \sum_{j=0}^\infty x_j \right)
  = T x.
  \]
  Therefore, $y \in T(B_X(0, 3))$ and thus $\bar{T(B_X(0, 1))}
  \subset T(B_X (0,3))$.

  Now for every $\epsilon > 0$, we have 
  $\frac{\epsilon}{3} \bar{T(B_X(0, 1))} \subset \frac{\epsilon}{3}
  T(B_X(0, 3)) = T(B_X(0, \epsilon))$. By assumption, there exists
  $\delta > 0$ such that $B_Y(0, \delta) \subset \bar{T(B_X(0, 1))}$.
  Therefore,
  \[
  B_Y \left( 0, \frac{\delta \epsilon}{3} \right) = \frac{\epsilon}{3} 
  B_Y(0, \delta) \subset \frac{\epsilon}{3} \bar{T(B_X(0, 1))}
  \subset T(B_X(0, \epsilon)).
  \]

  (3) $\implies$ (4). Let $U \subset X$ be open and $y \in T(U)$.
  There exists $x \in U$ such that $T x = y$. Since $U$ is open,
  there exists $\epsilon > 0$ such that $B_X(x, \epsilon)
  \subset U$. By assumption, there exists $\delta > 0$
  such that $B_Y(0, \delta) \subset T(B_X(0, \epsilon))$.
  It follows that 
  \[
  B_Y(y, \delta) = y + B_Y(0, \delta) \subset Tx + T(B_X(0, \epsilon)) 
  = T(x + B_X(0, \epsilon)) \subset T(U).
  \]
  Therefore, $T(U)$ is open and $T$ is an open map.

  (4) $\implies$ (5). Since $T$ is an open map, $T(B_X(0, 1))$ is 
  open. Also, $T(0) = 0$ so there exists $r > 0$ such that 
  $B_Y(0, r) \subset T(B_X(0, 1))$. Now let $y \in Y$. Then, 
  $\frac{r}{2 \norm{y}} y \in B_Y(0, r)$ and there exists 
  $x \in B_X(0, 1)$ such that $Tx = \frac{r}{2 \norm{y}} y$. 
  It follows that 
  \[
  T \left( \frac{2\norm{y}}{r} x \right) = y,
  \]
  and since $x \in B_X(0, 1)$,
  \[
  \norm{\frac{2 \norm{y}}{r} x} = \frac{2 \norm{y} \norm{x}}{r}
  < \frac{2}{r} \norm{y}.
  \]
  Letting $C = \frac{2}{r}$ completes the proof.

  (5) $\implies$ (1). Since for each $y \in Y$ there exists 
  $x \in X$ such that $T x = y$, $T$ is surjective.
\end{proof}

\subsubsection*{Linear homeomorphisms, norm equivalence, and closed graphs }
\begin{thm}
    Let $X$ and $Y$ be Banach spaces and suppose that $T \in \L(X,Y)$ is a bijection.  Prove that $T^{-1} \in \L(Y,X)$, and in particular $T$ is a linear (and thus bi-Lipschitz) homeomorphism.
\end{thm}

\begin{proof}
  Since $T \in \L(X, Y)$ is a bijection, $T$ is a surjection. 
  It follows that $T$ is an open map. In particular, for any 
  $U \subset X$ open, $T(U) = (T^{-1})^{-1}(U)$ is open. Therfore,
  $T^{-1}$ is continuous and thus $T$ is a linear homeomorphism.
\end{proof}

\begin{thm}
Let $X$ be a vector space that is complete when equipped 
with both of the norms $\norm{\cdot}_1$ and 
$\norm{\cdot}_2$.  Prove that if there exists a constant 
$C_1>0$ such that $\norm{x}_2 \le C_1 \norm{x}_1$ for all 
$x \in X$, then there exists a constant $C_0 >0$ such that 
$C_0 \norm{x}_1 \le \norm{x}_2 \le C_1 \norm{x}_1$ for all 
$x \in X$.    
\end{thm}

\begin{proof}
Let $T: X_1 \to X_2$, where $X_1$ and $X_2$
are $X$ equipped with norms $\norm{\cdot}_1$ and $\norm{\cdot}_2$,
respectively, be the identity map. Then for any $x \in X$ 
with $\norm{x}_1 = 1$, we have 
\[
\norm{T x}_2 = \norm{x}_2 \leq C_1 \norm{x}_1 = C_1.
\]
Therefore, $T \in \L(X_1, X_2)$. $T$ is also surjective.
Therefore, there exists a constant $C \geq 0$ such that 
each $\norm{x}_1 \leq C \norm{x}_2$. Hence, for each 
$x \in X$
\[
\frac{1}{C} \norm{x}_1 \leq \norm{x}_2 \leq C_1 \norm{x}_1.
\]
Letting $C_0 = \frac{1}{C}$ completes the proof.
\end{proof}

\begin{thm}
    Let $X$ and $Y$ be Banach spaces and let $T : X \to Y$ 
    be linear (just the algebraic condition).  
    Prove that the following are equivalent
\end{thm}
\begin{enumerate}
\item $T$ is continuous, i.e. $T \in \L(X;Y)$.
\item The graph of $T$, $\Gamma(T) = \{(x,Tx) : x \in X\} 
\subset X \times Y$, is closed in $X \times Y$, where 
$X \times Y$ is endowed with any of the usual $p$-norms.
\end{enumerate}

\begin{proof}
  (a) $\implies$ (b). Let $\left\{ (x_n, T x_n) \right\}$ be a convergent sequence 
  in $\Gamma(T)$. Since $X$ is Banach, $x_n \to x$ for some $x \in X$.
  Since $T \in \L(X ; Y)$, we have 
  \[
  \lim_{n \to \infty} T x_n = T \left( \lim_{n \to \infty} x_n \right) = T x.
  \]
  Therefore, $(x_n, T x_n) \to (x, T x) \in \Gamma(T)$, and thus 
  $\Gamma(T)$ is closed.

  (b) $\implies$ (a). Let $\pi_1 : \Gamma(T) \to X$ and 
  $\pi_2 : \Gamma(T) \to Y$ by 
  $\pi_1 (x, T x) = x$ and $\pi_2 (x, T x) = Tx$. Since 
  $\Gamma(T)$ is a closed in Banach space $Y$, 
  $\Gamma(T)$ is Banach space. It is clear that both $\pi_1$
  and $\pi_2$ are bounded linear maps. Moreover, $\pi_1$ is a 
  bijection. It follows that $S = \pi_1^{-1}$ is a bounded linear 
  map. Therefore, $T = \pi_2 \circ S$ is a bounded linear map. 
\end{proof}

\subsubsection*{Linear injections with closed range}
\begin{thm}
    Let $X$ and $Y$ be Banach spaces and $T \in \L(X,Y)$.  
    Prove the following are equivalent.
\begin{enumerate}
 \item $T$ is injective and $\ran(T)$ is closed.
 \item $T : X \to \ran(T)$ is a linear homeomorphism.
 \item There exists $C\ge 0$ such that $\norm{x}_X \le C 
 \norm{Tx}_Y$ for all $x \in X$.
\end{enumerate}
HINT: Prove that $(1) \implies (2) \implies (3) \implies 
(1)$.
\end{thm}

\begin{proof}
  (1) $\implies$ (2). If $T$ is injective and $\ran(T)$ is closed,
  then $\Gamma(T) = \left\{ (x, Tx) : x \in X \right\}$ is closed in 
  $X \times Y$. Therefore, $T : X \to \ran (T)$ is a bounded linear 
  map. Since $T$ is injective, this map is actually bijective from 
  $X$ to $\ran (T)$. Therefore, $T$ is a linaer homeomorphism. 

  (2) $\implies$ (3). Since $T$ is a bijective bounded linear map,
  from $X$ to $\ran(T)$. There exists a contant $C \geq 0$
  such that for each $y \in \ran(T)$ there exists 
  a unique $x \in X$ such that $T x = y$ and $\norm{x} \leq
  C \norm{y} = C \norm{T x}$. Since $T$ is a bijection, 
  $\norm{x} \leq C \norm{T x}$ for all $x \in X$. 

  (3) $\implies$ (1). Let $x \in X$ be such that $T x = 0$.
  It follows that $\norm{x} \leq C \norm{T x} = 0$. Therefore, 
  $x = 0$ and $T$ is injective. To show that 
  $\ran(T)$ is closed, consider a convergent sequence 
  $\left\{ y_n \right\} \subset \ran(T)$ with 
  $y_n = T x_n$. Since for any $n, m \in \N$ we have 
  \[
  \norm{x_n - x_m} \leq C \norm{T (x_n - x_m)} = C \norm{y_n - y_m},
  \]
  $\left\{ x_n \right\}$ is Cauchy. Since $X$ is Banach, 
  $x_n \to x$ for some $x \in X$. Therefore, for 
  all $n \in \N$ we have
  \[
  \norm{y_n - T x} = \norm{T (x_n - x)} \leq \norm{T} \norm{x_n - x},
  \]
  and $y_n \to T x$. Hence, $\ran(T)$ is closed and the proof is
  complete.
\end{proof}

\begin{thm}
  Let $X$ and $Y$ be Banach spaces over a common field. 
  Then, the following subsets of $\L(X; Y)$ are open:
  \begin{enumerate}
    \item $\left\{ T \in \L(X; Y) : \text{$T$ is surjective} \right\}$,
    \item $\left\{ T \in \L(X; Y) : \text{$T$ is injective with closed range} \right\}$,
    \item $\H(X; Y) = \left\{ T \in \L(X; Y) : \text{$T$ is a homeomorphism} \right\}$.
  \end{enumerate}
\end{thm}

\begin{proof}
\begin{enumerate}
\item Let $T \in \L(X; Y)$ be surjective. By open mapping 
theorem, there is $\delta > 0$ such that $B_Y(0, \delta) 
\subset TB_X(0, 1)$. By homogeneity we have 
$B_Y(0, r) \subset TB_X(0, \alpha r)$ for all $r > 0$ where 
$\alpha = \delta^{-1}$. Now let $S \in \L(X; Y)$ be such that 
$\norm{T - S} < \beta < (2\alpha)^{-1}$. Claim $S$ is surjective.

Let $y \in Y$, inductively construct sequences $\left\{ x_n \right\}$
and $\left\{ y_n \right\}$. First let $y_0 = y$. Then,
$\norm{y_0} \in B(0, 2\norm{y_0})$. Select 
$x_0 \in X$ be such that $T x_0 = y_0$ and $\norm{x_0} 
\leq 2 \alpha \norm{y_0}$. Suppose we have selected $y_i$,
$x_i$ for $0 \leq i \leq n$. Set $y_{n+1} = y_n - S x_n$
and select $x_{n+1}$ be such that $T x_{n+1} = y_{n+1}$
and $\norm{x_{n+1}} \leq 2\alpha \norm{y_{n+1}}$.
Then, we have 
\[
\norm{y_{n+1}} = \norm{T x_n - S x_n} \leq
\norm{T - S} \norm{x_n} < 2 \alpha\beta \norm{y_n}
\]
and 
\[
\norm{x_{n+1}} = 2 \alpha \norm{y_{n+1}} \leq
2 \alpha \norm{T - S} \norm{x_n} < 2 \alpha \beta \norm{x_n}.
\]
Note that $2 \alpha\beta < 1$ and $X$ is Banach, define 
\[
x = \sum_{n=0}^\infty x_n = \lim_{N \to \infty} 
\sum_{n = 0}^N x_n.
\]
Also note that $\lim_{n \to \infty} y_n = 0$. It follows 
that 
\[
S x = \sum_{n=0}^\infty S x_n 
= \sum_{n=0}^\infty (y_n - y_{n+1})
= y_0 - \lim_{n \to \infty} y_{n+1} = y.
\]
Therefore $S$ is surjective and the set of surjective 
bounded linear maps are open.

\item Suppose $T \in \L(X; Y)$ is injective with closed range.
Then, closed range theorem gives $C > 0$ such that 
$\norm{x} \leq C \norm{T x}$ for all $x \in X$. Now supose 
$S \in \L(X; Y)$ is such that $\norm{T - S} < (2C)^{-1}$.
Claim that $S$ is also injective with closed range. Indeed,
\[
\begin{aligned}
  \norm{x} &\leq C\norm{T x} \leq C \norm{Sx} + C \norm{(T-S)x} \\
  & \leq C \norm{Sx} + \frac{1}{2}\norm{x}.
\end{aligned}
\]
This shows that $\norm{x} \leq 2C \norm{Sx}$ for all $x \in X$.
By closed range theorem, $S$ is injective with closed range.
This implies that the set of injective bounded linear operator
with closed range is open.

\item This directly follows from 
\[
\H(X; Y) = \left\{ T \in \L(X; Y) : \text{$T$ is surjective} \right\} 
\cap \left\{ T \in \L(X; Y) : \text{$T$ is injective with closed range} \right\}.
\]
\end{enumerate}
\end{proof}

\begin{thm}
Let $X$ and $Y$ be Banach spaces over a common field. Then the following holds.
\begin{enumerate}
\item The sets 
\[
\L_R(X; Y) = \left\{ T \in \L(X; Y) : \text{there exists $S \in \L(Y; X)$ such that $ST = I_X$} \right\}
\]
and 
\[
\L_L(X; Y) = \left\{ T \in \L(X; Y) : \text{there exists $S \in \L(Y; X)$ such that $TS = I_Y$} \right\}
\]
are open.
\item The following inclusion holds: 
\[
\L_L(X; Y) \subset \left\{ T \in \L(X; Y) : \text{$T$ is surjective} \right\}
\]
and 
\[
\L_R(X; Y) \subset \left\{ T \in \L(X; Y) : \text{$T$ is injective with closed range} \right\}.
\]
\item The sets $\L_L(X; Y) \setminus \L_R(X; Y)$ and $\L_R(X; Y) \setminus \L_L(X; Y)$ are open.
\end{enumerate}
\end{thm}

\begin{proof}
\begin{enumerate}
\item Let $T_0 \in \L_R$ and $S_0 \in \L(Y; X)$ be such that 
$T_0 S_0 = I_Y$. Note that $I_X \in \H(X)$ and when $\norm{P} < 1$
for $P \in \L(X)$, we have $I_X + P \in \H(X)$. Suppose now 
$T \in \L(X; Y)$ and $\norm{T} < \norm{S_0}^{-1}$. It follows that 
$I_X + S_0 T \in \H(X)$. For such $T$, we then have 
\[
T_0 + T = T_0 (I_X + S_0 T).
\]
Also, 
\[
(T_0 + T) (I_X + S_0 T)^{-1} S_0 = T_0 (I_X + S_0 T) (I_X + S_0 T)^{-1} S_0
= T_0 S_0 = I_Y.
\]
Therefore, $T_0 + T \in \L_R$ for $T \in B(T_0, \norm{S_0}^{-1})$ 
and $\L_R$ is open.

Now let $T_0 \in \L_L$ and $S_0 \in \L(Y; X)$ be such that 
$S_0 T_0 = I_X$. Again, for $T \in \L(X; Y)$ with 
$\norm{T} < \norm{S_0}^{-1}$, we have 
\[
T_0 + T = (I_X + T S_0) T_0.
\]
and 
\[
S_0 (I_X + T S_0)^{-1} (T_0 + T) = I_X.
\]
Therefore, $\L_R$ is also open.

\item Let $T \in \L_R$ and $S \in \L(Y; X)$ be such that 
$TS = I_Y$. Then for any $y \in Y$ let $x = Sy$. It follows 
that $Tx = TSy = y$. Also, $\norm{x} \leq \norm{S} \norm{y}$
so the 4th item in open mapping theorem guarantees that $T$ is 
surjective. Hence, $\L_L \subset \left\{ T \in \L(X; Y) :  
\text{$T$ is surjective}\right\}$.

Now let $T \in \L_L$ and $S \in \L(Y; X)$ such that 
$ST = I_X$. Now for any $x \in X$, we have $\norm{x} 
= \norm{STx} \leq \norm{S} \norm{Tx}$. Then the closed 
range theorem guarantees that $T$ is injective with closed 
range. Hence, $\L_R \subset \left\{ T \in \L_R(X; Y) : 
\text{$T$ is injective with closed range} \right\}$.

\item ***TO-DO***
\end{enumerate}
\end{proof}

\section{Practice problems}
\subsubsection*{Problem 1}
Suppose $\omega: [0, \infty) \to [0, \infty]$ any function 
such that $\omega(x) = 0$ if and only if $x = 0$, 
$\omega$ continuous at 
$0$, and $\omega$ is nondecreasing. For $f: X \to Z$ define
\[
[f]_\omega = \sup\left\{ \frac{d(f(x),f(y))}{\omega(d(x,y))} :
x, y \in X, x \neq y \right\}
\]
and the space 
\[
C^{0, \omega} (X; Z) = \left\{ f: X \to Z \mid [f]_\omega < \infty \right\}.
\]
{
\newcommand{\cw}{{C^{0, \omega}}}
\newcommand{\cwb}{{C^{0, \omega}_b}}
\begin{enumerate}
  \item Prove that $\cw(X; Z) \subset C^0(X; Z)$.
  \begin{proof}
    Let $x \in X$ and $\epsilon > 0$. It follws that for 
    any $x \neq y$ we have 
    \[
    d(f(x), f(y)) \leq [f]_\omega \omega(d(x, y)).
    \]
    Since $\omega(0) = 0$, $\omega$ continuous
    at $0$, and $\omega$ is nondecreasing, we can find 
    $\delta > 0$ such that $0 \leq t < \delta$ implies 
    $0 \leq \omega(t) < \epsilon$. 
    Therefore,
    $d(x, y) < \delta$ implies $d(f(x), f(y)) < \epsilon [f]_{\omega}$.
    Since $[f]_{\omega} < \infty$, $f$ is continuous and 
    $\cw (X; Z) \subset C^0(X; Z)$.
  \end{proof}

  \item Suppose $Z$ Banach. Show that $\norm{f}_{\cw} = \norm{f}_{C^0} + [f]_\omega$
  is a norm on $\cwb(X; Z) = C^0_b(X; Z) \cap \cw(X; Z)$, and 
  that $\cwb(X;Z)$ is complete with respect to this norm.

  \begin{proof}
    It is easy to show that $\norm{\cdot}_{\cw}$ is indeed a
    norm on $\cwb(X; Z)$. Now we show that $\cwb(X; Z)$
    is complete with respect to this norm.
    Suppose $\left\{ f_n \right\} \subset \cwb$ Cauchy. Then 
    it is also Cauchy in $C^0_b$. Therefore there is $f \in C^0_b$
    such that $f_n \to f$ under $C^0$ norm. Remain to show 
    $[f - f_n]_\omega \to 0$. Let $x, y \in X$ and $x \neq y$
    and $m, n \geq N$ implies $[f_m - f_n]_\omega < \epsilon$.
    Then, 
    \[
    \begin{aligned}
      \frac{\norm{f_m(x) - f_m(y) - f_n(x) + f_n(y)}_Z}{\omega(d(x, y))}
      < \epsilon.
    \end{aligned}
    \]
    Take $m \to \infty$ and take supremum of all 
    $x, y \in X$ with $x \neq y$
    completes the proof.
  \end{proof}

  \item Suppose that $X$ is compact and $d \in \N$, show that 
  $B_{C^{0, \omega}(X; \R^d)} [0, 1] \subset C^0(X; \R^d)$
  is compact.

  \item Suppose $X$ compact and infinte, and $d \in \N$.
  Show that $B_{\cw}[0,1] \subset \cw$ is not compact.
  Conclude that $\id: (\cw, \norm{\cdot}_{C^0}) \to (\cw, \norm{\cdot}_{\cw})$
  is not continuous. Also conclude that $(\cw, \norm{\cdot}_{C^0})$ 
  is not complete.

  \item Another way to see this last fact is to first prove 
  $\cw(X; \R^d)$ is a strict subset of $C^0(X; \R^d)$. It is helpful 
  to study the sets $E_n = \left\{ f \in C^0(X; \R^d) : [f]_\omega \leq n \right\}$.
  Show that $\cw(X; \R)$ is dense in $C^0(X; \R)$. 
  Use this to show that $\cw(X; \R^d)$ is dense in $C^0(X; \R)$,
  and conclude $(\cw(X; \R^d), \norm{\cdot}_{C^0})$ 
  is not complete.
\end{enumerate}
}

\end{document}